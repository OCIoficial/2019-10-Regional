\documentclass{oci}
\usepackage[utf8]{inputenc}
\usepackage{lipsum}

\title{Reuniones}

\begin{document}
\begin{problemDescription}
  Sabina ha estado trabajando en una start-up durante los últimos años.
  Recientemente, Sabina ha visto su start-up crecer al punto de necesitar una
  nueva oficina.
  Uno de los aspectos más importantes para ella es
  asegurarse de que la nueva oficina tenga suficientes salas para todas las
  reuniones de la empresa.
  Como Sabina está muy ocupada siendo la dueña de la empresa, te ha pedido
  ayuda para resolver este problema.

  Cada reunión necesita tener una sala asignada, pero una sala no
  puede estar asignada a más de una reunión al mismo tiempo.
  Por lo tanto, una sala puede ser reutilizada si y solo si los tiempos de las
  reuniones asignadas a la sala no se intersectan.
  Puedes asumir que el tiempo en desocupar una oficina es despreciable y por
  lo tanto una reunión puede comenzar exactamente en el momento en que otra
  termina.

  Con el fin de ahorrar dinero, Sabina está buscando una oficina con la mínima
  cantidad de salas que le permita acomodar las reuniones de su empresa.
\end{problemDescription}

\begin{inputDescription}
    La primera línea de la entrada está compuesta de un solo entero $N$.
    Luego siguen $N$ líneas que describen el tiempo de las reuniones.
    La $i$-ésima línea contiene dos enteros, $p_i$ y $q_i$ ($p_i < q_i$),
    separados por un espacio, representando respectivamente el tiempo en que
    comienza y termina la $i$-ésima reunión.
\end{inputDescription}

\begin{outputDescription}
    La salida debe contener un único entero, correspondiente a la cantidad
    mínima de salas que debe tener la nueva oficina para poder alojar todas las
    reuniones.
\end{outputDescription}

\begin{scoreDescription}
  \subtask{10}
    Se probarán varios casos donde se cumplen las siguientes restricciones:
    \begin{itemize}
    \item $0 < N \leq 2$
    \item $0 \leq p_i< q_i \leq 10^3$
    \end{itemize}
  \newpage
  \subtask{10}
    Se probarán varios casos donde se cumplen las siguientes restricciones:
    \begin{itemize}
    \item $0 < N \leq 3$
    \item $0 \leq p_i< q_i \leq 10^3$
    \end{itemize}
  \subtask{30}
    Se probarán varios casos donde se cumplen las siguientes restricciones:
    \begin{itemize}
    \item $0 < N \leq 10^3$
    \item $0 \leq p_i< q_i \leq 10^3$
    \end{itemize}
  \subtask{50}
    Se probarán varios casos donde se cumplen las siguientes restricciones:
    \begin{itemize}
    \item $0 < N \leq 10^6$
    \item $0 \leq p_i< q_i \leq 10^6$
    \end{itemize}
\end{scoreDescription}

\begin{sampleDescription}
\sampleIO{sample-1}
\sampleIO{sample-2}
\end{sampleDescription}

\end{document}
