\documentclass{oci}
\usepackage[utf8]{inputenc}
\usepackage{lipsum}

\title{Problema Presidencial: Paltas Promedio}

\begin{document}
\begin{problemDescription}
En el último discurso presidencial, se explicó el incremento de la actividad criminal a nivel país en base al reciente alza del precio de la palta Hass. Esta correlación, que la presidencia denomina ICPC (Índice de Correlación entre Paltas y Crimen) llegaría supuestamente al 65\%. Ahora solo necesitan una forma de convencer a la población de que esa cifra completamente inventada es cierta.\\

\noindent Para ello, la Presidencia de la República solicitó a la Organización de Coincidencias Intencionales, también conocida como OCI, encontrar la mayor muestra de datos posible que valide esta estadística.\\

\noindent Dentro de los datos que la OCI tiene a disposición están los valores reales del ICPC en estos últimos $n$ días, y la misión por lo tanto es encontrar el mayor período de días consecutivos en que el promedio del ICPC alcance el valor deseado $t$.\\

\noindent Por ejemplo, si en los últimos 6 días el ICPC ha tenido los valores \texttt{0, 60, 100, 40, 25, 33}, y se desea mostrar a la gente que el ICPC es de $t=50\%$, se pueden mostrar solamente los primeros 4 valores: \texttt{0, 60, 100, 40}, ya que su promedio es exactamente 50\%.\\

\noindent La OCI ha descubierto que participas de olimpiadas de informática y por lo mismo piensa que eres la persona perfecta para su nuevo plan de manipulación social.\\
\end{problemDescription}

\begin{inputDescription}
\noindent Dos enteros $n$ y $t$ en una misma línea, separados por un espacio.\\

\noindent En la siguiente línea, $n$ enteros $A_i$ separados por espacios.
Los $A_i$ son los valores reales del ICPC proporcionados por la OCI.\\

\noindent Puedes asumir que las siguientes restricciones siempre se cumplen:

$1 \le n \le 10^6$

$0 \le t \le 100$

$0 \le A_i \le 100$
\end{inputDescription}

\begin{outputDescription}
\noindent Dos enteros $s$ y $\ell$ en una misma línea, separados por espacio, que describen el período de días consecutivos que se le mostrará a la población (un subsegmento del arreglo).\\

\noindent $s$ es el índice dentro de $A$ donde comienza el subsegmento. El índice $0$ corresponde al primer elemento.\\

\noindent $\ell$ es el largo de ese subsegmento.\\

\noindent Si hay dos o más subsegmentos del mismo largo que cumplen con los requisitos, se debe elegir el que comience primero (menor $s$).\\

\noindent Si no hay ningún subsegmento que cumpla con los requisitos, se debe imprimir \texttt{-1}.\\
\end{outputDescription}

\begin{scoreDescription}
  \subtask{10}
  Se probarán varios casos donde $n \le 250$.
 
  \subtask{20}
  Se probarán varios casos donde $n \le 5000$.
 
  \subtask{40}
  Se probarán varios casos donde $n \le 10^6$.
  
  $A$ estará ordenado de menor a mayor (o sea $A_i \le A_{i+1}$ para todo $i$)
 
  \subtask{30}
  Se probarán varios casos donde $n \le 10^6$.
  
  $A$ no estará necesariamente ordenado.
\end{scoreDescription}

\begin{sampleDescription}
\sampleIO{sample-1}
\sampleIO{sample-2}
\sampleIO{sample-3}
\end{sampleDescription}

\end{document}
